\documentclass[11pt]{article}
\usepackage{times}
\usepackage{amsfonts}
\usepackage{color}
\usepackage{multirow}
%%\usepackage{rotating}
\usepackage{url}
\usepackage{latexsym}
\pagenumbering{arabic}

\title{Location-Based Dynamic Social Networks}


\author{
   Vladimir Eidelman, Raul Guerra, and Jim Stevens\\
Department of Computer Science \\
University of Maryland, College Park\\
   %\affaddr{College Park, MD 20742}\\
 {\tt \{vlad,rguerra,jims}@cs.umd.edu}

}
\begin{document}

\maketitle 

\section{Introduction}


Imagine walking into a classroom or party, and having access to a dynamic social network with the other people in your immediate area, including desired profile information and real time information which is dependent on the venue. Motivated by the previous idea, the Proteus group implemented an application that generates dynamic networks that are location dependent for a user to immediately recognize people of interest or for a user to establish contact more easily with other users. Currently the application is a web application that runs on web browsers, which makes it extremely portable. However the application does need device specific information, the GPS location of the device running the application. This is not a problem because as the HTML5 standard becomes more pervasive, this application will be able to be deployed in any cellphone, PDA, iphones, etc. whose browser supports HTML5 geolocation capability.


Users access the application through a web interface. When users first log in into the application, they create an account with information they are going to share with other people and information that will be used to recognize other users that are of interest. A user's profile resembles an online business card, only containing information that the user does not mind sharing with other people, since the emphasis is not in sharing personal information but on networking.   To complement the profile information, the users is constantly updating his or her context information. The context information is used by the application to find other users of interest. The context information can range from a fixed list of the user's interests to more transient information like the user's location. The interface of the application interacts with a server to find other users in a location and pulling these users' context. The interface also utilizes Google Maps' interface to display the network of users logged in at a specific location. Through the Google Map interface, the user can see other users' real time location, can see their profile information, and interact with them. While using the application, users will enter and leave the dynamic social network all the time, depending on whether a person is considered to be ''present'' in the network or not. To determine whether a person is considered to be "present", the application will use GPS information from the user's browser.


\section{System Architecture}

%%diagram


\subsection{Database Setup}

%%database names and descriptions

\subsection{User Interaction}

%%user interface

\subsection{Google Mashup}

\subsection{Modularity}

%%how we can plug in/out fake users




\section{Simulation}

\subsection{User Generation}

\subsection{Controller}

%%daemon






\section{Intended Use}




The project's biggest value resides both in the services, it provides and the services it can be extended to provide, and the data it will produce once it is deployed. The application will produce a new type of social network data. While using the application, the users will be part of a complex social network which will be created and destroyed ad hoc in a short period of time. This will ellicit social interactions that do not happen in more rigid social networks like Facebook. This new type of data will raise questions in the area of Information Visualization, how can this rapidly changing data be best presented to a user? Machine Learning, how can this rapidly changing data be used by a learning system to infer information about the user? Can the application create surprise models based on this information. If the application becomes as widely used as other Social Networks, then more structural questions are raised. What other hardware should be added to a mobile device to extend the application's capabilities or to make the information handling faster? Networks, would other design structures (like peer-to-peer) improve the performance of the application?


\subsection{Graffiti/Tagging}

\subsection{Business Card/Profile}


\section{Future Work}

Currently the application only implements very general and simplified versions of the services previously described. In the current application, users can create an account with their profile information, and the only context information they provide is their real time location. Because of the limited context information, the user recommendation engine could not be implemented; there is not enough information to recommend one user over another. The Google Maps interface does show all the logged in users at a specific location moving in real time and allows a user to see other users' profile information and interact with them by writing in other users' walls. Future work would focus on the back end; implementing a recommendation engine;  making the application deployable; and making the application scalable.


By no means is the application limited only to facilitating social interactions. The application's capabilities can be extended to provide almost any service that is location dependent. 


OTHER

Furthermore the application's services can be complemented by any information captured by sensors currently in and in future mobile devices by capturing the context around the user. This data can be used to better inform the server about a user's surroundings and for the server to make better inferences about the user's surroundings and make suggestions to the user.


Improve the way the server recommends users


Export the interactions made in the application


A user can also keep track of who he or she talked to during an event, and after the event he or she can choose to keep a link through exporting it to a more permanent social networking ut Then, using profile information and algorithms yet to be developed, we can decide how best to present the other members of the current dynamic networks to the user, so that they can see that they do know someone at the party, or that their friend is right next door having lunch.




 \section{Conclusion}


\end{document}



